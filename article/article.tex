\documentclass[10pt,a4paper,onecolumn]{article}
% \usepackage[utf8]{inputenc}
\usepackage{marginnote}
\usepackage{graphicx}
\usepackage{xcolor}
\usepackage{authblk,etoolbox}
\usepackage{titlesec}
\usepackage{calc}
\usepackage{hyperref}
\hypersetup{breaklinks=true,
            bookmarks=true,
            pdfauthor={},
            pdftitle={},
            colorlinks=true,
            citecolor=blue,
            urlcolor=blue,
            linkcolor=blue,
            pdfborder={0 0 0}}
\urlstyle{same}
\usepackage{tcolorbox}
\usepackage{ragged2e}
\usepackage{fontspec}
\usepackage{fontawesome}
\usepackage{caption}
\usepackage{listings}
\lstnewenvironment{code}{\lstset{language=Haskell,basicstyle=\small\ttfamily}}{}



%\usepackage{fancyvrb}
%\VerbatimFootnotes
%\usepackage{graphicx}
%\usepackage{mdframed}
%\newmdenv[backgroundcolor=lightgray]{Shaded}


\usepackage{longtable,booktabs}

\usepackage[
  backend=biber,
%  style=alphabetic,
%  citestyle=numeric
]{biblatex}
\bibliography{article.bib}



% --- Macros ------------------------------------------------------------------
\renewcommand*{\bibfont}{\small \sffamily}

\definecolor{red}{HTML}{CF232B}
\newcommand{\ReScience}{Re{\bfseries \textcolor{red}{Science}}}

\newtcolorbox{rebox}
   {colback=blue!5!white, colframe=blue!40!white,
     boxrule=0.5pt, arc=2pt, fonttitle=\sffamily\scshape\bfseries,
     left=6pt, right=20pt, top=6pt, bottom=6pt}

\newtcolorbox{repobox}
   {colback=red, colframe=red!75!black,
     boxrule=0.5pt, arc=2pt, left=6pt, right=6pt, top=3pt, bottom=3pt}

% fix for pandoc 1.14     
\newcommand{\tightlist}{%
  \setlength{\itemsep}{1pt}\setlength{\parskip}{0pt}\setlength{\parsep}{0pt}}

% --- Style -------------------------------------------------------------------
\renewcommand*{\bibfont}{\small \sffamily}
\renewcommand{\captionfont}{\small\sffamily}
\renewcommand{\captionlabelfont}{\bfseries}

\makeatletter
\renewcommand\@biblabel[1]{{\bf #1.}}
\makeatother

% --- Page layout -------------------------------------------------------------
\usepackage[top=3.5cm, bottom=3cm, right=1.5cm, left=1.5cm,
            headheight=2.2cm, reversemp, includemp, marginparwidth=4.5cm]{geometry}

% --- Section/SubSection/SubSubSection ----------------------------------------
\titleformat{\section}
  {\normalfont\sffamily\Large\bfseries}
  {}{0pt}{}
\titleformat{\subsection}
  {\normalfont\sffamily\large\bfseries}
  {}{0pt}{}
\titleformat{\subsubsection}
  {\normalfont\sffamily\bfseries}
  {}{0pt}{}
\titleformat*{\paragraph}
  {\sffamily\normalsize}


% --- Header / Footer ---------------------------------------------------------
\usepackage{fancyhdr}
\pagestyle{fancy}
%\renewcommand{\headrulewidth}{0.50pt}
\renewcommand{\headrulewidth}{0pt}
\fancyhead[L]{\hspace{-1cm}\includegraphics[width=4.0cm]{rescience-logo.pdf}}
\fancyhead[C]{}
\fancyhead[R]{} 
\renewcommand{\footrulewidth}{0.25pt}

\fancyfoot[L]{\hypersetup{urlcolor=red}
              \sffamily \ReScience~$\vert$
              \href{http://rescience.github.io}{rescience.github.io}
              \hypersetup{urlcolor=blue}}
\fancyfoot[C]{\sffamily \thepage}
\fancyfoot[R]{\sffamily Sep 2015 $\vert$
                        Volume \textbf{1} $\vert$
                        Issue \textbf{1}}
\pagestyle{fancy}
\makeatletter
\let\ps@plain\ps@fancy
\fancyheadoffset[L]{4.5cm}
\fancyfootoffset[L]{4.5cm}

% --- Title / Authors ---------------------------------------------------------
% patch \maketitle so that it doesn't center
\patchcmd{\@maketitle}{center}{flushleft}{}{}
\patchcmd{\@maketitle}{center}{flushleft}{}{}
% patch \maketitle so that the font size for the title is normal
\patchcmd{\@maketitle}{\LARGE}{\LARGE\sffamily}{}{}
% patch the patch by authblk so that the author block is flush left
\def\maketitle{{%
  \renewenvironment{tabular}[2][]
    {\begin{flushleft}}
    {\end{flushleft}}
  \AB@maketitle}}
\makeatletter
\renewcommand\AB@affilsepx{ \protect\Affilfont}
%\renewcommand\AB@affilnote[1]{{\bfseries #1}\hspace{2pt}}
\renewcommand\AB@affilnote[1]{{\bfseries #1}\hspace{3pt}}
\makeatother
\renewcommand\Authfont{\sffamily\bfseries}
\renewcommand\Affilfont{\sffamily\small\mdseries}
\setlength{\affilsep}{1em}

\LetLtxMacro{\OldIncludegraphics}{\includegraphics}
\renewcommand{\includegraphics}[2][]{\OldIncludegraphics[width=12cm, #1]{#2}}


% --- Document ----------------------------------------------------------------
\title{[Re] Chaos in a long-term experiment with a plankton community}

    \usepackage{authblk}
                        \author[1]{Owen Petchey}
                    \author[1]{Marco Plebani}
                    \author[1]{Frank Pennekamp}
                            \affil[1]{Institute of Evolutionary Biology and Environmental Studies, University
of Zurich, Zurich, Switzerland}
            
\date{\vspace{-5mm}
      \sffamily \small \href{mailto:owen.petchey@ieu.uzh.ch}{owen.petchey@ieu.uzh.ch}}


\setlength\LTleft{0pt}
\setlength\LTright{0pt}


\begin{document}
\maketitle

\marginpar{
  %\hrule
  \sffamily\small
  %\vspace{2mm}
  {\bfseries Editor}\\
  Name Surname\\

  {\bfseries Reviewers}\\
        Name Surname\\
        Name Surname\\
  
  {\bfseries Received}  Sep, 1, 2015\\
  {\bfseries Accepted}  Sep, 1, 2015\\
  {\bfseries Published} Sep, 1, 2015\\

  {\bfseries Licence}   \href{http://creativecommons.org/licenses/by/4.0/}{CC-BY}

  \begin{flushleft}
  {\bfseries Competing Interests:}\\
  The authors have declared that no competing interests exist.
  \end{flushleft}

  \hrule
  \vspace{3mm}

  \hypersetup{urlcolor=white}
  
    \vspace{-1mm}
  \begin{repobox}
    \bfseries\normalsize
      \href{http://github.com/rescience/rescience-submission/article}{\faGithubAlt~Article repository}
  \end{repobox}
      \vspace{-1mm}
  \begin{repobox}
    \bfseries\normalsize
      \href{http://github.com/rescience/rescience-submission/code}{\faGithubAlt~Code repository}
  \end{repobox}
        \hypersetup{urlcolor=blue}
}

\begin{rebox}
\sffamily {\bfseries A reference implementation of}
\small
\begin{flushleft}
\begin{itemize}
    \item[→] Benincà, E., Huisman, J., Heerkloss, R., Jöhnk, K.D., Branco, P., Van
Nes, E.H., Scheffer, M. \& Ellner, S.P. (2008) Chaos in a long-term
experiment with a plankton community. Nature, 451, 822--825 DOI:
10.1038/nature06512
  \end{itemize}\par
\end{flushleft}
\end{rebox}


\section{Introduction}\label{introduction}

The original paper describes analyses of fluctuations in the abundance
of organisms in a plankton community derived from the Baltic Sea, housed
in a laboratory environment. The length of the time series (samples
every few days for 2,300 days) allowed for analyses revealing that the
observed dynamics exhibited characteristics consistent with chaos
produced by species interactions. The article concludes that stability
is not required for persistence of complex food webs, and that long-term
prediction of abundances may be fundamentally impossible. The
demonstration of chaotic dynamics and limited forecast horizons (sensu
\textcite{Petchey2015}) are important in the field of ecology, since the
ability to predict dynamics is an open question with considerable
applied importance \textcite{Petchey2015} \textcite{Mouquet2015}.

\section{Methods}\label{methods}

This reproduction started with the raw data (source given below) and
used information from the original paper, the supplementary information
\href{http://www.nature.com/nature/journal/v451/n7180/extref/nature06512-s1.pdf}{the
Supplement to the Nature paper}, and communications with Elisa Benincà
and Stephen Ellner. The latter provided code used to produce results in
the original paper, and its use in this reproduction is indicated below.

\subsection{The data}\label{the-data}

The data are available as an Excel file supplement to
\href{http://onlinelibrary.wiley.com/doi/10.1111/j.1461-0248.2009.01391.x/abstract}{an
Ecology Letters publication} \textcite{Beninca2009}. The Excel file
contains several datasheets. Two are particularly important, as they are
the source of the raw data (one contains original species abundances,
the one with the nutrient concentrations). We saved these two datasheets
as comma separated value text files. In the code associated with this
reproduction, these data files are read from the associated github
repository.

Another datasheet in the Ecology Letters supplement contains transformed
variables, though this was not used in this reproduction, as we
reproduced the described transformations and applied them to the raw
data.

We also received some data direct from Steve Ellner, see below for
details.

\subsection{Reproduction environment}\label{reproduction-environment}

The R language and environment for statistical computing and graphics
was used to make the reproduction. Additional R packages required are:

\begin{verbatim}
library(tidyr)
library(dplyr)
library(lubridate)
library(stringr)
library(ggplot2)
library(RCurl)
library(pracma)
library(oce)
library(tseriesChaos)
library(reshape2)
library(mgcv)
library(repmis)
\end{verbatim}

\section{Results}\label{results}

\subsection{Population dynamics}\label{population-dynamics}

The raw data show populations dynamics at least very similar to those in
figure 1b-g of the original publication (figure \ref{fig:dynamics}).

\begin{figure}[htbp]
\centering
\includegraphics{figures/unnamed-chunk-21-1}
\caption{\label{fig:dynamics}Observed population dynamics.}
\end{figure}

\subsection{Data transformation}\label{data-transformation}

Did our transformation give data matching the data we received from
Ellner, and that in the Ecology Letters supplement?

\subsection{Spectral analyses}\label{spectral-analyses}

\subsection{Correlations among species
abundances}\label{correlations-among-species-abundances}

\subsection{Lyapunov exponents by direct
method}\label{lyapunov-exponents-by-direct-method}

\subsection{Lyapunov exponents by indirect
method}\label{lyapunov-exponents-by-indirect-method}

\subsection{Predictability decay}\label{predictability-decay}

Only done for non-linear models (the original article also does this for
linear models).

\section{Conclusion}\label{conclusion}

Conclusion, at the very minimum, should indicate very clearly if you
were able to replicate original results. If it was not possible but you
found the reason why (error in the original results), you should exlain
it.

\begin{longtable}[c]{@{}llllll@{}}
\caption{\label{tbl:table}Table caption }\tabularnewline
\toprule
Heading 1 & & & Heading 2 & &\tabularnewline
\midrule
\endfirsthead
\toprule
Heading 1 & & & Heading 2 & &\tabularnewline
\midrule
\endhead
cell1 row1 & cell2 row 1 & cell3 row 1 & cell4 row 1 & cell5 row 1 &
cell6 row 1\tabularnewline
cell1 row2 & cell2 row 2 & cell3 row 2 & cell4 row 2 & cell5 row 2 &
cell6 row 2\tabularnewline
cell1 row3 & cell2 row 3 & cell3 row 3 & cell4 row 3 & cell5 row 3 &
cell6 row 3\tabularnewline
\bottomrule
\end{longtable}

A reference to table \ref{tbl:table}. A reference to figure
\ref{fig:logo}. A reference to equation \ref{eq:1}. A reference to
citation \textcite{markdown}.

\begin{figure}[htbp]
\centering
\includegraphics{rescience-logo.pdf}
\caption{\label{fig:logo}Figure caption}
\end{figure}

\begin{equation} A = \sqrt{\frac{B}{C}} \label{eq:1}\end{equation}

\section{Acknowledgements}\label{acknowledgements}

This reproduction was made as part of the Reproducible Research in
Ecology, Evolution, Behaviour, and Environmental Studies (RREEBES)
Course, lead by Owen Petchey at the University of Zurich. More
information about the course
\href{https://github.com/opetchey/RREEBES/blob/master/README.md}{here}
on github.

{\sffamily \small
  \printbibliography[title=References]
}
\end{document}
