\documentclass[10pt,a4paper,onecolumn]{article}
% \usepackage[utf8]{inputenc}
\usepackage{marginnote}
\usepackage{graphicx}
\usepackage{xcolor}
\usepackage{authblk,etoolbox}
\usepackage{titlesec}
\usepackage{calc}
\usepackage{hyperref}
\hypersetup{breaklinks=true,
            bookmarks=true,
            pdfauthor=
{
      Timothée Poisot,
  },
            pdftitle=
{
[Re] On the coexistence of specialists and generalists
},
            colorlinks=true,
            citecolor=blue,
            urlcolor=blue,
            linkcolor=blue,
            pdfborder={0 0 0}}
\urlstyle{same}
\usepackage{tcolorbox}
\usepackage{ragged2e}
\usepackage{fontspec}
\usepackage{fontawesome}
\usepackage{caption}
\usepackage{listings}
\lstnewenvironment{code}{\lstset{language=Haskell,basicstyle=\small\ttfamily}}{}



%\usepackage{fancyvrb}
%\VerbatimFootnotes
%\usepackage{graphicx}
%\usepackage{mdframed}
%\newmdenv[backgroundcolor=lightgray]{Shaded}


\usepackage{longtable,booktabs}

\usepackage[
  backend=biber,
%  style=alphabetic,
%  citestyle=numeric
]{biblatex}
\bibliography{bibliography.bib}



% --- Macros ------------------------------------------------------------------
\renewcommand*{\bibfont}{\small \sffamily}

\definecolor{red}{HTML}{CF232B}
\newcommand{\ReScience}{Re{\bfseries \textcolor{red}{Science}}}

\newtcolorbox{rebox}
   {colback=blue!5!white, colframe=blue!40!white,
     boxrule=0.5pt, arc=2pt, fonttitle=\sffamily\scshape\bfseries,
     left=6pt, right=20pt, top=6pt, bottom=6pt}

\newtcolorbox{repobox}
   {colback=red, colframe=red!75!black,
     boxrule=0.5pt, arc=2pt, left=6pt, right=6pt, top=3pt, bottom=3pt}

% fix for pandoc 1.14     
\newcommand{\tightlist}{%
  \setlength{\itemsep}{1pt}\setlength{\parskip}{0pt}\setlength{\parsep}{0pt}}

% --- Style -------------------------------------------------------------------
\renewcommand*{\bibfont}{\small \sffamily}
\renewcommand{\captionfont}{\small\sffamily}
\renewcommand{\captionlabelfont}{\bfseries}

\makeatletter
\renewcommand\@biblabel[1]{{\bf #1.}}
\makeatother

% --- Page layout -------------------------------------------------------------
\usepackage[top=3.5cm, bottom=3cm, right=1.5cm, left=1.5cm,
            headheight=2.2cm, reversemp, includemp, marginparwidth=4.5cm]{geometry}

% --- Section/SubSection/SubSubSection ----------------------------------------
\titleformat{\section}
  {\normalfont\sffamily\Large\bfseries}
  {}{0pt}{}
\titleformat{\subsection}
  {\normalfont\sffamily\large\bfseries}
  {}{0pt}{}
\titleformat{\subsubsection}
  {\normalfont\sffamily\bfseries}
  {}{0pt}{}
\titleformat*{\paragraph}
  {\sffamily\normalsize}


% --- Header / Footer ---------------------------------------------------------
\usepackage{fancyhdr}
\pagestyle{fancy}
%\renewcommand{\headrulewidth}{0.50pt}
\renewcommand{\headrulewidth}{0pt}
\fancyhead[L]{\hspace{-1cm}\includegraphics[width=4.0cm]{rescience-logo.pdf}}
\fancyhead[C]{}
\fancyhead[R]{} 
\renewcommand{\footrulewidth}{0.25pt}

\fancyfoot[L]{\hypersetup{urlcolor=red}
              \sffamily \ReScience~$\vert$
              \href{http://rescience.github.io}{rescience.github.io}
              \hypersetup{urlcolor=blue}}
\fancyfoot[C]{\sffamily 1 - \thepage}
\fancyfoot[R]{\sffamily Sep 2015 $\vert$
                        Volume \textbf{1} $\vert$
                        Issue \textbf{1}}
\pagestyle{fancy}
\makeatletter
\let\ps@plain\ps@fancy
\fancyheadoffset[L]{4.5cm}
\fancyfootoffset[L]{4.5cm}

% --- Title / Authors ---------------------------------------------------------
% patch \maketitle so that it doesn't center
\patchcmd{\@maketitle}{center}{flushleft}{}{}
\patchcmd{\@maketitle}{center}{flushleft}{}{}
% patch \maketitle so that the font size for the title is normal
\patchcmd{\@maketitle}{\LARGE}{\LARGE\sffamily}{}{}
% patch the patch by authblk so that the author block is flush left
\def\maketitle{{%
  \renewenvironment{tabular}[2][]
    {\begin{flushleft}}
    {\end{flushleft}}
  \AB@maketitle}}
\makeatletter
\renewcommand\AB@affilsepx{ \protect\Affilfont}
%\renewcommand\AB@affilnote[1]{{\bfseries #1}\hspace{2pt}}
\renewcommand\AB@affilnote[1]{{\bfseries #1}\hspace{3pt}}
\makeatother
\renewcommand\Authfont{\sffamily\bfseries}
\renewcommand\Affilfont{\sffamily\small\mdseries}
\setlength{\affilsep}{1em}

\LetLtxMacro{\OldIncludegraphics}{\includegraphics}
\renewcommand{\includegraphics}[2][]{\OldIncludegraphics[width=12cm, #1]{#2}}


% --- Document ----------------------------------------------------------------
\title{[Re] On the coexistence of specialists and generalists}

    \usepackage{authblk}
                        \author[1]{Timothée Poisot}
                            \affil[1]{Département de Sciences Biologiques, Université de Montréal, Montréal,
QC, Canada}
            
\date{\vspace{-5mm}
      \sffamily \small \href{mailto:timothee.poisot@umontreal.ca}{timothee.poisot@umontreal.ca}}


\setlength\LTleft{0pt}
\setlength\LTright{0pt}


\begin{document}
\maketitle

\marginpar{
  %\hrule
  \sffamily\small
  %\vspace{2mm}
  {\bfseries Editor}\\
  Name Surname\\

  {\bfseries Reviewers}\\
        Name Surname\\
        Name Surname\\
  
  {\bfseries Received}  Sep, 1, 2015\\
  {\bfseries Accepted}  Sep, 1, 2015\\
  {\bfseries Published} Sep, 1, 2015\\

  {\bfseries Licence}   \href{http://creativecommons.org/licenses/by/4.0/}{CC-BY}

  \begin{flushleft}
  {\bfseries Competing Interests:}\\
  The authors have declared that no competing interests exist.
  \end{flushleft}

  \hrule
  \vspace{3mm}

  \hypersetup{urlcolor=white}
  
    \vspace{-1mm}
  \begin{repobox}
    \bfseries\normalsize
      \href{http://github.com/rescience/rescience-submission/article}{\faGithubAlt~Article repository}
  \end{repobox}
      \vspace{-1mm}
  \begin{repobox}
    \bfseries\normalsize
      \href{http://github.com/rescience/rescience-submission/code}{\faGithubAlt~Code repository}
  \end{repobox}
        \hypersetup{urlcolor=blue}
}

\begin{rebox}
\sffamily {\bfseries A reference implementation of}
\small
\begin{flushleft}
\begin{itemize}
    \item[→] On the coexistence of specialists and generalists, David Sloan Wilson
and Jin Yoshimura, The American Naturalist 144:4, 692-707, 1994.
  \end{itemize}\par
\end{flushleft}
\end{rebox}


\section{Introduction}\label{introduction}

The coexistence of specialists and generalist within ecological
communities is a long-standing question. \textcite{wils94csg} have
suggested that this coexistence can be understood when examined in the
light of (i) differential fitness loss associated to specialism, (ii)
active habitat selection, (iii) negative density dependence due to
competition, and (iv) stochastic changes in habitat quality, that allow
combinations of species to persist even though coexistence would not be
possible in a purely deterministic world. Here I propose an
implementation of this model in \emph{Julia} \autocite{beza17jfa}, and
show that it is able to reproduce most figures from the original
manuscript.

\section{Methods}\label{methods}

The \textcite{wils94csg} model describes three species across two
patches of habitat. Species 1 is a specialist of habitat 1, species 2 is
a specialist of habitat 2, and species 3 is a generalist. This results
in the maximum density that these species can reach in both habitats:

\begin{equation}
\mathbf{K} = \begin{bmatrix}
  K_1 & aK_1 \\
  a_K2 & K_2 \\
  bK_1 & b_K2
\end{bmatrix} \,.
\end{equation}

In this matrix, \(K_1\) is the quality of habitat 1, \(K_2\) is the
quality of habitat 2, \(a\) is the fitness cost of the specialist in its
non-optimal environment, and \(b\) is the fitness cost of generalism.
Note that \(1 > b > a > 0\).

Species distribute themselves across habitats in a way that minimizes
the negative effect of other species on their fitness. This is modelled
by each species having a value \(p_i\), which is the proportion of its
species choosing habitat 1. Values of \(\mathbf{p}\) are found by
measuring the negative density effect of each species in each habitat:

\begin{equation}
D_{l1} = \frac{\sum_{i\in l,m,m}p_iN_i}{K_{l1}}
\end{equation}

and

\begin{equation}
D_{l2} = \frac{\sum_{i\in l,m,m}(1-p_i)N_i}{K_{l1}} \,.
\end{equation}

The value of \(p_l\) for which \(D_{l1}=D_{l2}\) is

\begin{equation}
p_l = - \frac{(K_{l1}+K_{l2})(N_mp_m+N_np_n)-K_{l1}(N_l+N_m+N_n)}{N_l(K_{l1}+K_{l2})} \,.
\end{equation}

We fix \(p_m\) and \(p_n\), and find the value of \(p_l\), while
enforcing the constraint of \(0 \leq p_l \leq 1\). Repeating this
procedure a few times for the different species yields the optimal
values of \(\mathbf{p}\); we can measure the density of individuals in
both habitats. Before we do so, there is a proportion \(g\) of
individuals that select habitats at random. Given a total population
size of \(N_i\), there are \(N_i(g/2)\) individuals will go to either
habitat, and \(N_i(1-g)p_i\) will pick habitat 1. With this information,
we can write the matrix describing habitat selection:

\begin{equation}
\mathbf{N} = \begin{bmatrix}
  N_1(\frac{g}{2}+(1-g)p_1) & N_1(\frac{g}{2}+(1-g)(1-p_1))\\
  N_2(\frac{g}{2}+(1-g)p_2) & N_2(\frac{g}{2}+(1-g)(1-p_2))\\
  N_3(\frac{g}{2}+(1-g)p_3) & N_3(\frac{g}{2}+(1-g)(1-p_3))
\end{bmatrix} \,.
\end{equation}

Finally, the fitness of every species in each habitat is given by

\begin{equation}
W_{ij} = \text{exp}\left[r\left(1-\frac{N_{i1}+N_{i2}+N_{i3}}{K_{ij}}\right)\right] \,,
\end{equation}

where \(r\) is the growth rate (assumed equal). The population size at
the next timestep is simply given by

\begin{equation}
\mathbf{N}_{t+1} = \mathbf{W}\odot \mathbf{N}_{t} \,,
\end{equation}

where \(\odot\) is the element-wise multiplication.

\section{Results}\label{results}

Original sources were not available, and no attempts were made to
contact the authors. For some non-stochastic situations, it is possible
to calculate expected values by hand. The original manuscript does
provide some of these values, and they were used to test this
implementation. Figure 2A and 2B in the original manuscript provide a
good diagnostic value, as they are based on non-stochastic situations.
In \textcite{fig:02},

\begin{figure}
\centering
\includegraphics{figure02.pdf}
\caption{Population dynamics and habitat preference of the generalist
alone (left), and following invasion by the two specialists at the
generalist equilibrium (right).}\label{fig:02}
\end{figure}

Results should be compared with original results and you have to explain
why you think they are the same or why they may differ (qualitative
result vs quantitative result). Note that it is not necessary to redo
all the original analysis of the results.

\section{Conclusion}\label{conclusion}

Conclusion, at the very minimum, should indicate very clearly if you
were able to replicate original results. If it was not possible but you
found the reason why (error in the original results), you should exlain
it.

\begin{longtable}[]{@{}llllll@{}}
\caption{Table caption \{\#tbl:table\}}\tabularnewline
\toprule
\begin{minipage}[b]{0.16\columnwidth}\raggedright\strut
Heading 1\strut
\end{minipage} & \begin{minipage}[b]{0.16\columnwidth}\raggedright\strut
\strut
\end{minipage} & \begin{minipage}[b]{0.16\columnwidth}\raggedright\strut
\strut
\end{minipage} & \begin{minipage}[b]{0.16\columnwidth}\raggedright\strut
Heading 2\strut
\end{minipage} & \begin{minipage}[b]{0.16\columnwidth}\raggedright\strut
\strut
\end{minipage} & \begin{minipage}[b]{0.16\columnwidth}\raggedright\strut
\strut
\end{minipage}\tabularnewline
\midrule
\endfirsthead
\toprule
\begin{minipage}[b]{0.16\columnwidth}\raggedright\strut
Heading 1\strut
\end{minipage} & \begin{minipage}[b]{0.16\columnwidth}\raggedright\strut
\strut
\end{minipage} & \begin{minipage}[b]{0.16\columnwidth}\raggedright\strut
\strut
\end{minipage} & \begin{minipage}[b]{0.16\columnwidth}\raggedright\strut
Heading 2\strut
\end{minipage} & \begin{minipage}[b]{0.16\columnwidth}\raggedright\strut
\strut
\end{minipage} & \begin{minipage}[b]{0.16\columnwidth}\raggedright\strut
\strut
\end{minipage}\tabularnewline
\midrule
\endhead
cell1 row1 & cell2 row 1 & cell3 row 1 & cell4 row 1 & cell5 row 1 &
cell6 row 1\tabularnewline
cell1 row2 & cell2 row 2 & cell3 row 2 & cell4 row 2 & cell5 row 2 &
cell6 row 2\tabularnewline
cell1 row3 & cell2 row 3 & cell3 row 3 & cell4 row 3 & cell5 row 3 &
cell6 row 3\tabularnewline
\bottomrule
\end{longtable}

A reference to table \textcite{tbl:table}. A reference to figure
\textcite{fig:logo}. A reference to equation \textcite{eq:1}. A
reference to citation \textcite{markdown}.

\begin{figure}
\centering
\includegraphics{rescience-logo.pdf}
\caption{Figure caption}\label{fig:logo}
\end{figure}

\[ A = \sqrt{\frac{B}{C}} \] \{\#eq:1\}

{\sffamily \small
  \printbibliography[title=References]
}
\end{document}
