\documentclass[10pt,a4paper,onecolumn]{article}
% \usepackage[utf8]{inputenc}
\usepackage{marginnote}
\usepackage{graphicx}
\usepackage{xcolor}
\usepackage{authblk,etoolbox}
\usepackage{titlesec}
\usepackage{calc}
\usepackage{hyperref}
\hypersetup{breaklinks=true,
            bookmarks=true,
            pdfauthor=
{
      Georgios Detorakis,
  },
            pdftitle=
{
[Re] Multiple dynamical modes of thalamic relay neurons: rhythmic bursting
        and intermittent phase-locking
},
            colorlinks=true,
            citecolor=blue,
            urlcolor=blue,
            linkcolor=blue,
            pdfborder={0 0 0}}
\urlstyle{same}
\usepackage{tcolorbox}
\usepackage{ragged2e}
\usepackage{fontspec}
\usepackage{fontawesome}
\usepackage{caption}
\usepackage{listings}
\usepackage{float,lscape}
\usepackage{xcolor,colortbl}
\usepackage{amssymb,amsmath,array,amsthm}
\usepackage{marvosym}
\usepackage{array}
\usepackage[export]{adjustbox}

\newcommand{\Rm}[1]{\mathrm{#1}}

\makeatletter
\newcommand{\thickhline}{%
    \noalign {\ifnum 0=`}\fi \hrule height 1pt
    \futurelet \reserved@a \@xhline
}
\newcolumntype{"}{@{\hskip\tabcolsep\vrule width 1pt\hskip\tabcolsep}}
\makeatother

\definecolor{Gray}{gray}{0.75}
\definecolor{LightGray}{gray}{0.95}
\lstnewenvironment{code}{\lstset{language=Haskell,basicstyle=\small\ttfamily}}{}



%\usepackage{fancyvrb}
%\VerbatimFootnotes
%\usepackage{graphicx}
%\usepackage{mdframed}
%\newmdenv[backgroundcolor=lightgray]{Shaded}


\usepackage{longtable,booktabs}

\usepackage[
  backend=biber,
%  style=alphabetic,
%  citestyle=numeric
]{biblatex}
\bibliography{detorakis-2016.bib}



% --- Macros ------------------------------------------------------------------
\renewcommand*{\bibfont}{\small \sffamily}

\definecolor{red}{HTML}{CF232B}
\newcommand{\ReScience}{Re{\bfseries \textcolor{red}{Science}}}

\newtcolorbox{rebox}
   {colback=blue!5!white, colframe=blue!40!white,
     boxrule=0.5pt, arc=2pt, fonttitle=\sffamily\scshape\bfseries,
     left=6pt, right=20pt, top=6pt, bottom=6pt}

\newtcolorbox{repobox}
   {colback=red, colframe=red!75!black,
     boxrule=0.5pt, arc=2pt, left=6pt, right=6pt, top=3pt, bottom=3pt}

% fix for pandoc 1.14     
\newcommand{\tightlist}{%
  \setlength{\itemsep}{1pt}\setlength{\parskip}{0pt}\setlength{\parsep}{0pt}}

% --- Style -------------------------------------------------------------------
\renewcommand*{\bibfont}{\small \sffamily}
\renewcommand{\captionfont}{\small\sffamily}
\renewcommand{\captionlabelfont}{\bfseries}

\makeatletter
\renewcommand\@biblabel[1]{{\bf #1.}}
\makeatother

% --- Page layout -------------------------------------------------------------
\usepackage[top=3.5cm, bottom=3cm, right=1.5cm, left=1.5cm,
            headheight=2.2cm, reversemp, includemp, marginparwidth=4.5cm]{geometry}

% --- Section/SubSection/SubSubSection ----------------------------------------
\titleformat{\section}
  {\normalfont\sffamily\Large\bfseries}
  {}{0pt}{}
\titleformat{\subsection}
  {\normalfont\sffamily\large\bfseries}
  {}{0pt}{}
\titleformat{\subsubsection}
  {\normalfont\sffamily\bfseries}
  {}{0pt}{}
\titleformat*{\paragraph}
  {\sffamily\normalsize}


% --- Header / Footer ---------------------------------------------------------
\usepackage{fancyhdr}
\pagestyle{fancy}
%\renewcommand{\headrulewidth}{0.50pt}
\renewcommand{\headrulewidth}{0pt}
\fancyhead[L]{\hspace{-1cm}\includegraphics[width=4.0cm]{rescience-logo.pdf}}
\fancyhead[C]{}
\fancyhead[R]{} 
\renewcommand{\footrulewidth}{0.25pt}

\fancyfoot[L]{\hypersetup{urlcolor=red}
              \sffamily \ReScience~$\vert$
              \href{http://rescience.github.io}{rescience.github.io}
              \hypersetup{urlcolor=blue}}
\fancyfoot[C]{\sffamily \thepage}
\fancyfoot[R]{\sffamily Sep 2015 $\vert$
                        Volume \textbf{1} $\vert$
                        Issue \textbf{1}}
\pagestyle{fancy}
\makeatletter
\let\ps@plain\ps@fancy
\fancyheadoffset[L]{4.5cm}
\fancyfootoffset[L]{4.5cm}

% --- Title / Authors ---------------------------------------------------------
% patch \maketitle so that it doesn't center
\patchcmd{\@maketitle}{center}{flushleft}{}{}
\patchcmd{\@maketitle}{center}{flushleft}{}{}
% patch \maketitle so that the font size for the title is normal
\patchcmd{\@maketitle}{\LARGE}{\LARGE\sffamily}{}{}
% patch the patch by authblk so that the author block is flush left
\def\maketitle{{%
  \renewenvironment{tabular}[2][]
    {\begin{flushleft}}
    {\end{flushleft}}
  \AB@maketitle}}
\makeatletter
\renewcommand\AB@affilsepx{ \protect\Affilfont}
%\renewcommand\AB@affilnote[1]{{\bfseries #1}\hspace{2pt}}
\renewcommand\AB@affilnote[1]{{\bfseries #1}\hspace{3pt}}
\makeatother
\renewcommand\Authfont{\sffamily\bfseries}
\renewcommand\Affilfont{\sffamily\small\mdseries}
\setlength{\affilsep}{1em}

\LetLtxMacro{\OldIncludegraphics}{\includegraphics}
\renewcommand{\includegraphics}[2][]{\OldIncludegraphics[width=12cm, #1]{#2}}


% --- Document ----------------------------------------------------------------
\title{[Re] Multiple dynamical modes of thalamic relay neurons: 
        rhythmic bursting and intermittent phase-locking}

    \usepackage{authblk}
                        \author[1]{Georgios Detorakis}
                        \affil[1]{ Department of Cognitive Sciences, UC Irvine, CA, USA}
            
\date{\vspace{-5mm}
    \sffamily \small \href{mailto:gdetorak@uci.edu}{gdetorak@uci.edu}}


\setlength\LTleft{0pt}
\setlength\LTright{0pt}


\begin{document}
\maketitle

\marginpar{
  %\hrule
  \sffamily\small
  %\vspace{2mm}
  {\bfseries Editor}\\
  Name Surname\\

  {\bfseries Reviewers}\\
        Name Surname\\
        Name Surname\\
  
  {\bfseries Received}  Sep, 1, 2015\\
  {\bfseries Accepted}  Sep, 1, 2015\\
  {\bfseries Published} Sep, 1, 2015\\

  {\bfseries Licence}   \href{http://creativecommons.org/licenses/by/4.0/}{CC-BY}

  \begin{flushleft}
  {\bfseries Competing Interests:}\\
  The authors have declared that no competing interests exist.
  \end{flushleft}

  \hrule
  \vspace{3mm}

  \hypersetup{urlcolor=white}
  
    \vspace{-1mm}
  \begin{repobox}
    \bfseries\normalsize
      \href{http://github.com/rescience/rescience-submission/article}{\faGithubAlt~Article repository}
  \end{repobox}
      \vspace{-1mm}
  \begin{repobox}
    \bfseries\normalsize
      \href{http://github.com/rescience/rescience-submission/code}{\faGithubAlt~Code repository}
  \end{repobox}
        \hypersetup{urlcolor=blue}
}

\begin{rebox}
\sffamily {\bfseries A reference implementation of}
\small
\begin{flushleft}
\begin{itemize}
    \item[→] \emph{Multiple dynamical modes of thalamic relay neurons: rhythmic
        bursting and intermittent phase-locking}, Wang, X-J, Neuroscience,
        59(1), pg. 21--31, 1994.
  \end{itemize}\par
\end{flushleft}
\end{rebox}


\section{Introduction}\label{introduction}

This work introduces a reference implementation of a neuron model for
thalamocortical relay neurons, proposed by X-J Wang, \cite{wang:1994}.
The model is conductance-based and takes advantage of an interplay between
a T-type calcium current and a non-specific cation sag current and thus, it
is able to generate spindle and delta rhythms. Another feature of this model
is the presence of an intermittent phase-locking phenomenon where action 
potentials of sodium take place in a non-periodic manner, despite the fact
that they are phase-locked to the periodic input current. Finally, the model
is capable of generating tonic spike patterns. The source code of
reference implementation is written in Python (Numpy, Scipy, Matplotlib,
and Scikit-image).


\section{Methods}\label{methods}

In this section, a detailed description of the model is given following 
the paradigm of Nordlie et al, \cite{nordlie:2009}. To this end description of
the model, equations, parameters, and inputs are given in the form of tables. 

Table~\ref{Table:1} provides a description of the model, Table~\ref{Table:2}
provides information about simulations duration and temporal integration time
steps, Table~\ref{Table:3} gives a glimpse of the input signals used in this
work. Table~\ref{Table:4} introduces the equations of the model and finally 
Table~\ref{Table:5} summarizes all the parameters for each figure (simulation).
The neuron model is conductance-based consisting of four differential equations
describing the dynamics of membrane potential and the kinetics of a T-Type
calcium current, a Sag current channel and a Potassium channel. The rest
currents are described by algebraic equations.
%%
\begin{table}[!htbp]
    \centering
    \begin{tabular}{ll}
        \thickhline
        \multicolumn{2}{c}{Model Summary} \\\thickhline
        \rowcolor{Gray}
        Populations  & No population -- one neuron model \\\rowcolor{LightGray}
        Topology     & -- \\ \rowcolor{Gray}
        Connectivity & -- \\ \rowcolor{LightGray}
        Neuron Model & Hodgkin-Huxley conductance-based \\\rowcolor{Gray}
        Channel Models & \\ \rowcolor{LightGray}
        Synapse Model & -- \\ \rowcolor{Gray}
        Plasticity & -- \\ \rowcolor{LightGray}
        Input & Constant current or periodic rectangular pulses \\\rowcolor{Gray}
        Measurements & Membrane potential, channels activation, phase plane \\
        \thickhline
    \end{tabular}
    \caption{{\bfseries \sffamily Summary of the model}} 
    \label{Table:1}
\end{table}
 %%  
The reference implementation has been done in a Python class (Python $3.5.1$)
along with Numpy (version $1.10.4$), Scipy (version $0.17.0$), Matplotlib
($1.5.1$) and Scikit-image (version $0.12.3$). The numerical integration
has been done using the \emph{ode} method of Scipy \emph{integrate} package.
Three different methods have been tested in this work (\emph{dopri5}, 
\emph{Adams}, \emph{BDF}, \cite{ascher:1998}). \emph{Dopri5} is the closest
numerical method to the one used by \cite{wang:1994} (fifth-order adaptive 
size Runge-Kutta method). \emph{BDF} and \emph{Adams} provide similar numerical
results as the first one, but they are faster. In the current implementation
of \cite{wang:1994} the user has the option to choose one of the three methods
at the stage of class instantiating. All three aforementioned methods have
been tested by comparing, spike times, amplitudes and the coefficient
of variation using as threshold value $0\, \Rm{mV}$. Spike events were
overlapping, and $CV \simeq 3.5$ for all three methods. A difference found 
in the amplitude of spikes generated by the three methods. Figure~\ref{Fig:1}
shows two histograms of error distribution (error is defined to be the
absolute difference of spike amplitude between two methods). Top histogram 
depicts a distribution of differences in amplitude between \emph{dopri5} 
and \emph{Adams} methods (gray color). The bottom histogram illustrates the 
distribution of amplitude differences between \emph{dopri5} and \emph{BDF}
methods (red color). In the rest of the article all the results have been 
produced by using the \emph{Adams} method.
%%
\begin{table}[!htbp]
    \centering
    \begin{tabular}{ccc}
        \thickhline
        \multicolumn{3}{c}{Simulation Time} \\ \thickhline
        Figure & Simulation Time ($\Rm{s}$) & Integration Step ($\Rm{ms}$) \\ 
        \rowcolor{LightGray}
        $1$ & $6$ & $0.05$ \\ \rowcolor{Gray}
        $2$ & $15\times period$, $5$ & $0.05$ \\ \rowcolor{LightGray}
        $3$ & $6$ & $0.05$ \\ \rowcolor{Gray} 
        $4$ & $1.5$, $1$, $1$ & $0.05$ \\ \rowcolor{LightGray} 
        $5$ & $5$ & $0.05$  \\ \rowcolor{Gray} 
        $6$ & $5$ & $0.05$  \\ \rowcolor{LightGray} 
        $7$ & $40$, $20$ & $0.05$  \\ \thickhline
    \end{tabular}
    \caption{{\bfseries \sffamily Simulations Time}}
    \label{Table:2}
\end{table}
%%  

\begin{figure}[!htbp]
    \centering
    \includegraphics[width=0.6\textwidth]{figs/Figure1.pdf}
    \caption{{\bfseries \sffamily Amplitude difference distribution} 
    {\bfseries \sffamily Top histogram} depicts the distribution of differences
    in spikes amplitudes between \emph{dopri5} and \emph{Adams} integration
        methods of Scipy \emph{integrate} package (black color). 
    {\bfseries \sffamily Bottom histogram} illustrates the distribution of
    differences in spikes amplitudes between \emph{dopri5} and \emph{BDF}
    methods (red color). In both cases the number of bins is $30$. 
    }
    \label{Fig:1}
\end{figure}

All simulations ran on a Dell OptiPlex $7040$, equipped with a sixth
generation i$7$ processor, $8\, \Rm{GB}$ of physical memory and running Arch
Linux.
The total execution time of all simulations was $526$ minutes and the peak
consumed memory was $465\, \Rm{MB}$\footnote{Python memory profiler used 
    (\url{https://pypi.python.org/pypi/memory_profiler}).}\@. 
All the parameters used in this work are given in Table~\ref{Table:5}. 
%%
\begin{table}[!htbp]
    \centering
    \begin{tabular}{llcccc}
        \thickhline
        \multicolumn{6}{c}{Input} \\\thickhline
        Figure  & Type & Form & \parbox[t]{1.5cm}{Frequency
            ($\frac{1}{P_0}$, $\Rm{Hz}$)} & 
            \parbox[t]{1.5cm}{Duration ($p$, $\Rm{ms}$)} &
            \parbox[t]{1.5cm}{Amplitude ($\mu \Rm{A} / \Rm{cm}^2$)} \\
        \thickhline \rowcolor{LightGray}
        Figure~$1$ & Constant & 
            \includegraphics[width=0.1\textwidth]{figs/const.pdf} &
            -- & -- & \parbox[c]{0.3cm}{$-0.6$}
            \\\rowcolor{Gray}
        Figure~$2$ & Periodic & 
            \includegraphics[width=0.1\textwidth]{figs/pulse.pdf} &
            \parbox[t]{1.1cm}{$5$, $10$} &
            \parbox[t]{1.1cm}{$10$, $40$} &
            \parbox[c]{0.3cm}{$-1.0$}
            \\\rowcolor{Gray}
        Figure~$3$ & Constant  & 
        \includegraphics[width=0.1\textwidth]{figs/const.pdf} &
        -- & -- & \parbox[c]{0.3cm}{$+3.0$ \\ $3$ $0.0$ $-0.45$ $-0.455$
            $-0.47$ $-0.55$ $-0.6$ $-0.8$ $-1.3$ $-1.4$ $-2.1$} 
        \\\rowcolor{LightGray}
        Figure~$4$ & Pulse/Constant  & 
            \includegraphics[width=0.1\textwidth]{figs/const.pdf} &
            -- & \parbox[c]{0.3cm}{100} & \parbox[c]{0.3cm}{$-1.25$ $0.25$ $-0.47$}
        \\\rowcolor{Gray}
        Figure~$5$ & Constant  & 
            \includegraphics[width=0.1\textwidth]{figs/const.pdf} &
            -- & -- & \parbox[c]{0.3cm}{$-0.95$}
        \\\rowcolor{LightGray}
        Figure~$6$ & Constant  & 
            \includegraphics[width=0.1\textwidth]{figs/const.pdf} &
            -- & -- & \parbox[c]{0.3cm}{$[-2,0]$}
        \\\rowcolor{Gray}
        Figure~$7$ & Constant  & 
            \includegraphics[width=0.1\textwidth]{figs/const.pdf} &
            -- & -- & \parbox[c]{0.3cm}{$[-2,0]$} \\
        \thickhline
    \end{tabular}
    \caption{{\bfseries \sffamily Description of the applied current
        $I_{\text{app}}$ }}
    \label{Table:3}
\end{table}
%%  

%%
\begin{table}[!htbp]
    \centering
    \begin{tabular}{p{1.5cm}ll}
        \thickhline
        \multicolumn{2}{c}{Neuron Model} \\\thickhline
        \rowcolor{Gray}
        Name  &  Thalamocortical relay neuron  \\ \rowcolor{LightGray}
        Type  &  Conductance-based neuron  \\ \rowcolor{Gray}
        Membrane Potential & $
            \begin{aligned}
                C_{\Rm{m}} \frac{dV(t)}{dt} &= -I_{\Rm{T}} - I_{\Rm{h}} - 
                I_{\Rm{Na}} - I_{\Rm{K}} - I_{\Rm{Na(P)}} - I_{\Rm{L}} +
                I_{\text{app}}
              \end{aligned}$ \\ \rowcolor{LightGray}
        T-Type Calcium Current ($I_T$) & $
            \begin{aligned}
                I_{\Rm{T}} &= g_{\Rm{T}} \cdot s^3_{\infty}(V)\cdot h \cdot
                (V-V_{\Rm{Ca}}) \\
                s_{\infty}(V) &= \frac{1}{1 + \exp(-\frac{V+65}{7.8})} \\
                \frac{dh(t)}{dt} &= \phi_h \frac{h_{\infty}(V) - h}{\tau_h(V)} \\
                h_{\infty}(V) &= \frac{1}{1 + \exp(\frac{V-\theta_h}{k_h})} \\
                \tau_h(V) &= h_{\infty}\exp(\frac{V+162.3}{17.8}) + 20
            \end{aligned} $
            \\ \rowcolor{Gray}
        Sag Current ($I_h$) & $
            \begin{aligned}
                I_h &= g_h \cdot H^2 \cdot (V-V_h) \\
                H_{\infty}(V) &= \frac{1}{1 + \exp(\frac{V+69}{7.1})} \\
                \frac{dH(t)}{dt} &= \phi_H \frac{H_{\infty}(V) - H}{\tau_H(V)} 
            \end{aligned} $
            \\ \rowcolor{LightGray} 
        Hodgkin-Huxley Currents ($I_K$) and ($I_{Na}$) & $
            \begin{aligned}
                I_{\Rm{K}} &= g_K \cdot n^4 \cdot (V - V_{\Rm{K}}) \\
                \frac{dn(t)}{dt} &= \phi_n \frac{n_{\infty}(V) - n(t)}{\tau_n(V)} \\
                n_{\infty}(\sigma_{\Rm{K}}, V) &=
                \frac{\alpha_n(\sigma_{\Rm{K}}, V)}
                {\alpha_n(\sigma_{\Rm{K}}, V) + \beta_n(\sigma_{\Rm{K}}, V)} \\
                \tau_n(\sigma_{\Rm{K}}, V) &= \frac{1}{\alpha_n(\sigma_{\Rm{K}}, V)
                + \beta_n(\sigma_{\Rm{K}}, V)} \\
                \alpha_n(\sigma_{\Rm{K}}, V) &= 
                \frac{-0.01 (V + 45.7 - \sigma_{\Rm{K}})}{\exp(-0.1(V + 45.7 - \sigma_{\Rm{K}})) - 1} \\ 
                \beta_n(\sigma_{\Rm{K}}, V) &= 0.125 \exp(-\frac{V + 55.7 -
                    \sigma_{\Rm{K}}}{80}) \\
                I_{\Rm{Na}} &= g_{\Rm{Na}} \cdot m^3_{\infty}(\sigma_{\Rm{Na}}, V)
                \cdot (0.85 - n) \cdot (V - V_{\Rm{Na}}) \\
                m_{\infty}(V) &= \frac{\alpha_m(\sigma_{\Rm{Na}}, V)}
                {\alpha_m(\sigma_{\Rm{Na}}, V) + \beta_m(\sigma_{\Rm{Na}}, V)} \\
                \alpha_m(\sigma_{\Rm{Na}}, V) &= -0.1 \frac{V + 29.7 -
                    \sigma_{\Rm{Na}}}
                    {\exp(-0.1(V + 54.7 - \sigma{\Rm{Na}})) - 1} \\
                    \beta_m(\sigma_{\Rm{Na}}, V) &=
                    4\exp(-\frac{V+54.7-\sigma_{\Rm{Na}}}{18})
            \end{aligned} $ \\ \rowcolor{Gray} 
        Persistent Sodium Currents ($I_{Na(P)}$) & $
                \begin{aligned}
                    I_{\Rm{Na(P)}} &= g_{\Rm{Na(P)}} \cdot
                    m^3_{\infty}(\sigma_{\Rm{Na(P)}},
                    V) \cdot (V - V_{\Rm{Na}}) \\
                \end{aligned} $
            \\ \rowcolor{LightGray}
        Leak Current ($I_L$) &  $
            \begin{aligned}
                I_L &= g_L \cdot (V - V_L) \\
            \end{aligned} $ \\ \thickhline
    \end{tabular}
    \caption{{\bfseries \sffamily Description of the neuron model}} 
    \label{Table:4}
\end{table}
%%

%%
\begin{landscape}
\begin{table}[!htbp]
    \centering
    \begin{tabular}{cccccccc}
        \thickhline
        \multicolumn{8}{c}{Model Parameters} \\\thickhline
        Figure & $V_0\, (\Rm{mV})$ & $g_T\, (\Rm{mS/cm^2})$ & $\theta_h\,
        (\Rm{mV})$ & $k_h\, (\Rm{mV^{-1}})$ & $\sigma_{Na}\, (\Rm{mV})$ &
        $g_L\, (\Rm{mS/cm^2})$ & $V_L\, (\Rm{mV})$ \\\rowcolor{LightGray}
        \thickhline
        $1$ & $-74$ & $1$ & $-79$ & $-5$ & $6$ & $0.12$ & $-70$ \\\rowcolor{Gray}
        $2$ & $-74$ & $1$ & $-81$ & $6.25$ & $3$ & $0.1$ & $-72$ \\\rowcolor{LightGray}
        $3$ & $-74$ & $0.3$ & $-79$ & $5$ & $6$ & $0.12$ & $-70$ \\\rowcolor{Gray}
        $4$ & $-72$/$-64$ & $1.0$ & $-79$ & $5$ & $6$ & $0.12$ & $-70$ \\\rowcolor{LightGray}
        $5$ & $-74$ & $0.3$ & $-75$ & $5$ & $6$ & $0.08$ & $-70$ \\\rowcolor{Gray}
        $6$ & $-74$ & $1$/$0.7$ & $-79$ & $5$ & $6$ & $0.1$/$0.04$ & $-70$ \\\rowcolor{LightGray}
        $7$ & $-72$ & $0.3$/$0.25$ & $-81$ & $6.25$ & $3$ & $0.1$ & $-72$ \\
        \thickhline
        \multicolumn{8}{l}{Common Parameters} \\\rowcolor{Gray}
        \multicolumn{8}{c}{$C_m = 1\, \Rm{\mu F/cm^2}$, $\phi_h = 2$,
        $V_{\Rm{Ca}} = 120\, \Rm{mV}$, $\phi_H = 1$, $g_h = 0.04\,
        \Rm{mS/cm^2}$, $V_h = -40\, \Rm{mV}$, $g_{\Rm{K}} = 30\, \Rm{mS/cm^2}$,
        $V_{\Rm{K}} = -80\, \Rm{mV}$, $g_{\Rm{Ca}} = 42\, \Rm{mS/cm^2}$} \\\rowcolor{Gray}
        \multicolumn{8}{c}{$V_{\Rm{Ca}} = 55\, \Rm{mV}$, $\phi_{n} = 28.5$,
        $\sigma_{\Rm{K}} = 10\, \Rm{mV}$, $V_{\Rm{Na(P)}} = 55\, \Rm{mV}$,
        $\sigma_{\Rm{Na(P)}} = -5\, \Rm{mV}$, $g_{\Rm{Na(P)}} = 9 \,
        \Rm{mS/cm^2}$}
        \\\thickhline
    \end{tabular}
    \caption{\bfseries \sffamily Simulation Parameters}
    \label{Table:5}
\end{table}
\end{landscape}
%%

\section{Results}\label{results}

We simulated the model described in Table~\ref{Table:4} using the parameters
given in Table~\ref{Table:5} and the corresponding input (see 
Table~\ref{Table:3}). First, we examined what is the response of the reference 
implementation to rhythmic hyperpolarization. In \cite{wang:1994} this is 
illustrated in Fig~$1$\footnote{From now and then original article's figures
    will be referred as Fig}. 
Thus, we applied a periodic current pulse of $-1\,\Rm{\mu A/cm^2}$ amplitude at
several different frequencies ($\frac{1}{P_0}$ is the frequency in $\Rm{Hz}$
and $P_0$ is the corresponding period in $\Rm{ms}$) ranging from 
$0.1\,\Rm{Hz}$ to $15\,\Rm{Hz}$ with a resolution (discretization) of $100$ 
points. The same number of samples used for discretizing the duration of the
pulse for each frequency. 

Upper panel shows two different simulations of the reference implementation at
specific frequencies ($5,\, 0.5\, \Rm{Hz}$) and ratios $p/P_0$ ($0.6, 0.6$),
respectively (following the upper panels of Fig~$1$ in \cite{wang:1994}).
These panels are comparable to the ones presented in \cite{wang:1994} and the 
timescales are exactly the same ($400\, \Rm{ms}$ for the left sub-panel and
$4\, \Rm{s}$ for the second one). However, numbers regarding pulse ON duration
($p$) in Fig~$1$ of \cite{wang:1994} are not correct. For instance, take the
frequency to be $\frac{1}{P_0} = 5\, \Rm{Hz}$ and $\frac{p}{P_0} = 0.6$ then
the pulse ON duration must be $p = 120\, \Rm{ms}$ and not $160\, \Rm{ms}$. 

Bottom panel illustrates the total results of the $100 \times 100$ simulations
we ran.  In order to create this figure, we counted the number of spikes
generated per pulse, both supra-threshold and sub-threshold, and then we 
computed the mean (this procedure is not described in a solid way in
\cite{wang:1994}, however it is described in \cite{mccormick:1990}).  
This figure is different from \cite{wang:1994}, where the shaded area that 
contains an average number of spikes below $1.0$ (but not zero) is larger
than in \cite{wang:1994}. Unfortunately, information regarding Fig~$1$
is limited in \cite{wang:1994} in order to be able to follow a specific recipe
and try to reproduce exactly the same figure. 

%%
\begin{figure}[!htbp]
    \centering
    \includegraphics[width=0.6\textwidth]{figs/Figure2.pdf}
    \caption{{\bfseries \sffamily Responses to rhythmic hyperpolarizations.}
        We stimulate the model with a periodic pulse with varying frequency
        $\frac{1}{P_0}$ in $[0.1, 15]\, \Rm{Hz}$ and ON pulse duration $p$. 
        {\bfseries \sffamily Upper panels} Show the results of two specific
        simulations out of the $100 \times 100$ we ran in total. The blue 
        curve illustrates a case where bursts of four spike take place when
        frequency is $\frac{1}{P_0} = 5\, \Rm{Hz}$ and $\frac{p}{P_0} = 0.6$.
        Black 
        curve depicts a case of two spikes bursts. In this case the frequency
        is $\frac{1}{P_0} = 0.5\, \Rm{Hz}$ and $\frac{p}{P_0} = 0.6$.
        {\bfseries \sffamily Bottom panel} shows a range of frequencies
        $\frac{1}{P_0}$ from $0.1\, \Rm{Hz}$ to $15\, \Rm{Hz}$ versus the ratio
        $\frac{p}{P_0}$ in range $[0, 1]$. Numbers indicate the average number
        of spikes per pulse. }
    \label{Fig:2}
\end{figure}
%%

Then we tested the transition from subthreshold to bursting oscillations via 
chaos. Therefore, we used a steady current varying only its amplitude keeping
all the other parameters fixed. We used eight different values (exactly the
same as in \cite{wang:1994}),
$$I_{\text{app}} =
3.0, 0.0, -0.45, -0.455, -0.47, -0.55, -0.6, -0.8, -1.3, -1.4, -2.0\, 
\Rm{\mu A /cm^2}.$$ Results are depicted in Figure~\ref{Fig:3}. This figure
corresponds to Fig~$3$ of \cite{wang:1994}. In this case, there was no
difference between the reference implementation and the original one. 
%%
\begin{figure}[!htbp]
    \centering
    \includegraphics[width=1.\textwidth]{figs/Figure3.pdf}
    \caption{{\bfseries \sffamily Dynamic behavior of neuron.} Here we
    simulate the model with parameters given in Table~\ref{Table:5}. The
    only parameter that varies from panel to panel in this figure is the
    external current $I_{\text{ext}}$. These results correspond to Fig~$3$ 
    of \cite{wang:1994}.}
    \label{Fig:3}
\end{figure}
%%

The next simulation we performed reproduces results related to hysteresis as
in Fig~$4$ of \cite{wang:1994}. First, we run a simulation where the input 
current $I_{ext}$ varied from $-0.433\, \Rm{\mu A / cm^2}$ to $-0.55 \mu A /cm^2$.
At every iteration, we detected if there is subthreshold and/or suprathreshold 
activity in the membrane voltage trace. Thus we created the hysteresis diagram
given in Figure~\ref{Fig:4}, top panel. 
%%
\begin{figure}[!htbp]
    \centering
    \includegraphics[width=0.6\textwidth]{figs/Figure4.pdf}
    \caption{{\bfseries \sffamily Hysteresis near the transition from the
    subthreshold to bursting oscillation.} {\bfseries \sffamily First panel:}
    Hysteresis diagram indicates a coexistence of bursting and subthreshold 
    activity. {\bfseries \sffamily Second panel:} Bistability example
    shows how the same system can produce either subthreshold activity (black
    curve) or burst activity (gray curve). {\bfseries \sffamily Third 
    panel:} shows the activation of $H$ for the same simulation as in the upper
    panel. {\bfseries  \sffamily Fourth panel:} Input to the model for
    this simulation is a brief current pulse at $0.25\, \Rm{\mu A /cm^2}$
    (black curve) and $-1.2\, \Rm{\mu A/cm^2}$ (gray curve) followed by a
    steady current at $-0.47\, \Rm{\mu A/cm^2}$ for both simulations. }
    \label{Fig:4}
\end{figure}
%%
In addition, we tested the bistability of the current model using the same 
protocol as in \cite{wang:1994}. According to that protocol, we first apply 
a brief pulse (for $100\, \Rm{ms}$) of $0.25\, \Rm{\mu A /cm^2}$ and
$-1.2\, \Rm{\mu A/cm^2}$ followed by a steady current at $-0.47\, 
\Rm{\mu A /cm^2}$. This causes the state of the model to switch from a purely 
subthreshold activity pattern to a mixed sub- and suprathreshold activity
pattern. The results are given in Figure~\ref{Fig:4}, right panel. 

Another interesting behavior of the model is the development of a ``spiral''
chaos (see Fig~$6$ in \cite{wang:1994}). The reference implementation is 
capable of generating similar behavior as Figure~\ref{Fig:5} shows. In order
to get these results we applied a constant external current with an amplitude
of $-0.95\, \Rm{\mu A / cm^2}$. 
%%
\begin{figure}[!htbp]
    \centering
    \includegraphics[width=.6\textwidth]{figs/Figure5.pdf}
    \caption{{\bfseries \sffamily ``Spiral Chaos''.} In this Figure
    we show that the model is capable of generating ``spiral chaos''
    as it has been shown in Fig~$6$ in \cite{wang:1994}.
    The external current in this simulation is constant ($-0.95\Rm{\mu A/cm^2}$). 
    {\bfseries \sffamily Top panel} shows the phase portrait of the 
    membrane potential and the Sag current ($V$ vs $h$).
    {\bfseries \sffamily Middle panel} illustrates the membrane potential ($V$)
    and, the {\bfseries \sffamily bottom panel} shows the $h$ current over time.}
    \label{Fig:5}
\end{figure}
%%

The next simulation investigates how the bursting frequency is affected by 
the injected external current (see Fig~$7$ in \cite{wang:1994}, two bursting
modes). To address this issue we captured data\footnote{Data are available in
    the accompanying github repository of the present article.} values for
the external current from Fig~$7$ of the original article (dots in Fig $7$,
pg. $27$ in \cite{wang:1994}) using a software called 
PlotDigitizer\footnote{\url{(http://plotdigitizer.sourceforge.net/})}.
The results from our simulations are illustrated in Figure~\ref{Fig:6}. Black
curve represents the frequency in $\Rm{Hz}$ and the blue one the period in 
$\Rm{sec}$. Black dots indicate frequency and open circles the corresponding
period (reference implementation), respectively. Cyan crosses show the original
data taken from \cite{wang:1994} and magenta pentagons the corresponding 
period. The original points and the reference implementation share a similar 
shape (and smoothness). However, there is a slight divergence for a few of 
the points. 
%%
\begin{figure}[!htbp]
    \includegraphics[width=1.0\textwidth, left]{figs/Figure6.pdf}
    \caption{{\bfseries \sffamily Frequency and period versus steady input 
    current.} This figure illustrates how the external current (ranging
    from $0$ to $-2\, \Rm{\mu A/cm^2}$ affects the bursts frequency (bursting
    mode) of the model. Black curve indicates frequency (black dots) in
    $\Rm{Hz}$ and blue curve period (white circle discs) in $\Rm{sec}$. 
    Cyan crosses (frequency) and magenta pentagons (period) indicate original
    data taken from \cite{wang:1994}.}
    \label{Fig:6}
\end{figure}
%%

The last simulation is related to Fig~$2$ of \cite{wang:1994} and to an
``intermittent'' phase-locking phenomenon. A periodic pulse with frequency 
$10\,\Rm{Hz}$ and ratio $\frac{p}{P_0} = 0.8$ is applied to the model as
external input current. The response of the model is registered and sub- and
supra-threshold spikes are counted. We applied six different current amplitudes
and two different values for the T-type calcium conductance. 
$I_{app} = -1.4, -1.5, -1.6\, \Rm{\mu A /cm^2}$ and $g_T = 0.3\, \Rm{mS/cm^2}$
$I_{app} = -1.2, -1.5, -1.8\, \Rm{\mu A/cm^2}$ and  $g_T = 0.25\,
\Rm{mS/cm^2}$. Figure~\ref{Fig:7} shows $2$ seconds of membrane potential 
along with symbolic patterns of $0$ (dark circles) and $1$ (green line
segments). Results given in Figure~\ref{Fig:7} are similar to the ones of
Fig~$2$A and Fig~$2$B in \cite{wang:1994}.
%%
\begin{figure}[!htbp]
    \includegraphics[width=1.1\textwidth]{figs/Figure7.pdf}
    \caption{{\bfseries \sffamily Symbolic patterns.} Six different simulations
    are illustrated here ($I_{app} = -1.4, -1.5 -1.6 \, \Rm{\mu A/cm^2}$
    with $g_T = 0.3\, \Rm{mS / cm^2}$) and ($I_{app} = -1.2, -1.5, -1.8\,
    \Rm{\mu A / cm^2}$ with $g_T = 0.25\, \Rm{mS / cm^2}$). Symbolic patterns
    are illustrated above membrane potential traces as green line segments -- 
    suprathreshold spikes and dark circles -- subthreshold spikes. Red curve 
    indicates the input periodic pulses.}
    \label{Fig:7}
\end{figure}
%%

\section{Conclusion}\label{conclusion}

A conductance-based model for relay thalamocortical neurons proposed 
by \cite{wang:1994} was implemented in Python. The model tested 
thoroughly in several examples taken from the original article. 
In general, the original model was easy to be implemented since all the
equations and the most of the parameters (except the initial time step of the
integration method) are given. The reference implementation results are similar
to the original ones except maybe of Figure~\ref{2} (Fig~$1$ of
\cite{wang:1994}), where we found some divergence from the original results.
Furthermore, not any other implementation of \cite{wang:1994} found in order
to be compared with the current reference implementation. 

{\sffamily \small
  \printbibliography[title=References]
}
\end{document}
