\documentclass[10pt,a4paper,onecolumn]{article}
% \usepackage[utf8]{inputenc}
\usepackage{marginnote}
\usepackage{graphicx}
\usepackage{xcolor}
\usepackage{authblk,etoolbox}
\usepackage{titlesec}
\usepackage{calc}
\usepackage{hyperref}
\hypersetup{breaklinks=true,
            bookmarks=true,
            pdfauthor=
{
      Owen Petchey,
      Marco Plebani,
      Frank Pennekamp,
  },
            pdftitle=
{
[Re] Chaos in a long-term experiment with a plankton community
},
            colorlinks=true,
            citecolor=blue,
            urlcolor=blue,
            linkcolor=blue,
            pdfborder={0 0 0}}
\urlstyle{same}
\usepackage{tcolorbox}
\usepackage{ragged2e}
\usepackage{fontspec}
\usepackage{fontawesome}
\usepackage{caption}
\usepackage{listings}
\lstnewenvironment{code}{\lstset{language=Haskell,basicstyle=\small\ttfamily}}{}



%\usepackage{fancyvrb}
%\VerbatimFootnotes
%\usepackage{graphicx}
%\usepackage{mdframed}
%\newmdenv[backgroundcolor=lightgray]{Shaded}


\usepackage{longtable,booktabs}

\usepackage[
  backend=biber,
%  style=alphabetic,
%  citestyle=numeric
]{biblatex}
\bibliography{article.bib}



% --- Macros ------------------------------------------------------------------
\renewcommand*{\bibfont}{\small \sffamily}

\definecolor{red}{HTML}{CF232B}
\newcommand{\ReScience}{Re{\bfseries \textcolor{red}{Science}}}

\newtcolorbox{rebox}
   {colback=blue!5!white, colframe=blue!40!white,
     boxrule=0.5pt, arc=2pt, fonttitle=\sffamily\scshape\bfseries,
     left=6pt, right=20pt, top=6pt, bottom=6pt}

\newtcolorbox{repobox}
   {colback=red, colframe=red!75!black,
     boxrule=0.5pt, arc=2pt, left=6pt, right=6pt, top=3pt, bottom=3pt}

% fix for pandoc 1.14     
\newcommand{\tightlist}{%
  \setlength{\itemsep}{1pt}\setlength{\parskip}{0pt}\setlength{\parsep}{0pt}}

% --- Style -------------------------------------------------------------------
\renewcommand*{\bibfont}{\small \sffamily}
\renewcommand{\captionfont}{\small\sffamily}
\renewcommand{\captionlabelfont}{\bfseries}

\makeatletter
\renewcommand\@biblabel[1]{{\bf #1.}}
\makeatother

% --- Page layout -------------------------------------------------------------
\usepackage[top=3.5cm, bottom=3cm, right=1.5cm, left=1.5cm,
            headheight=2.2cm, reversemp, includemp, marginparwidth=4.5cm]{geometry}

% --- Section/SubSection/SubSubSection ----------------------------------------
\titleformat{\section}
  {\normalfont\sffamily\Large\bfseries}
  {}{0pt}{}
\titleformat{\subsection}
  {\normalfont\sffamily\large\bfseries}
  {}{0pt}{}
\titleformat{\subsubsection}
  {\normalfont\sffamily\bfseries}
  {}{0pt}{}
\titleformat*{\paragraph}
  {\sffamily\normalsize}


% --- Header / Footer ---------------------------------------------------------
\usepackage{fancyhdr}
\pagestyle{fancy}
%\renewcommand{\headrulewidth}{0.50pt}
\renewcommand{\headrulewidth}{0pt}
\fancyhead[L]{\hspace{-1cm}\includegraphics[width=4.0cm]{rescience-logo.pdf}}
\fancyhead[C]{}
\fancyhead[R]{} 
\renewcommand{\footrulewidth}{0.25pt}

\fancyfoot[L]{\hypersetup{urlcolor=red}
              \sffamily \ReScience~$\vert$
              \href{http://rescience.github.io}{rescience.github.io}
              \hypersetup{urlcolor=blue}}
\fancyfoot[C]{\sffamily \thepage}
\fancyfoot[R]{\sffamily Sep 2015 $\vert$
                        Volume \textbf{1} $\vert$
                        Issue \textbf{1}}
\pagestyle{fancy}
\makeatletter
\let\ps@plain\ps@fancy
\fancyheadoffset[L]{4.5cm}
\fancyfootoffset[L]{4.5cm}

% --- Title / Authors ---------------------------------------------------------
% patch \maketitle so that it doesn't center
\patchcmd{\@maketitle}{center}{flushleft}{}{}
\patchcmd{\@maketitle}{center}{flushleft}{}{}
% patch \maketitle so that the font size for the title is normal
\patchcmd{\@maketitle}{\LARGE}{\LARGE\sffamily}{}{}
% patch the patch by authblk so that the author block is flush left
\def\maketitle{{%
  \renewenvironment{tabular}[2][]
    {\begin{flushleft}}
    {\end{flushleft}}
  \AB@maketitle}}
\makeatletter
\renewcommand\AB@affilsepx{ \protect\Affilfont}
%\renewcommand\AB@affilnote[1]{{\bfseries #1}\hspace{2pt}}
\renewcommand\AB@affilnote[1]{{\bfseries #1}\hspace{3pt}}
\makeatother
\renewcommand\Authfont{\sffamily\bfseries}
\renewcommand\Affilfont{\sffamily\small\mdseries}
\setlength{\affilsep}{1em}

\LetLtxMacro{\OldIncludegraphics}{\includegraphics}
\renewcommand{\includegraphics}[2][]{\OldIncludegraphics[width=12cm, #1]{#2}}


% --- Document ----------------------------------------------------------------
\title{[Re] Chaos in a long-term experiment with a plankton community}

    \usepackage{authblk}
                        \author[1]{Owen Petchey}
                    \author[1]{Marco Plebani}
                    \author[1]{Frank Pennekamp}
                            \affil[1]{Institute of Evolutionary Biology and Environmental Studies, University
of Zurich, Zurich, Switzerland}
            
\date{\vspace{-5mm}
      \sffamily \small \href{mailto:owen.petchey@ieu.uzh.ch}{owen.petchey@ieu.uzh.ch}}


\setlength\LTleft{0pt}
\setlength\LTright{0pt}


\begin{document}
\maketitle

\marginpar{
  %\hrule
  \sffamily\small
  %\vspace{2mm}
  {\bfseries Editor}\\
  Name Surname\\

  {\bfseries Reviewers}\\
        Name Surname\\
        Name Surname\\
  
  {\bfseries Received}  Sep, 1, 2015\\
  {\bfseries Accepted}  Sep, 1, 2015\\
  {\bfseries Published} Sep, 1, 2015\\

  {\bfseries Licence}   \href{http://creativecommons.org/licenses/by/4.0/}{CC-BY}

  \begin{flushleft}
  {\bfseries Competing Interests:}\\
  The authors have declared that no competing interests exist.
  \end{flushleft}

  \hrule
  \vspace{3mm}

  \hypersetup{urlcolor=white}
  
    \vspace{-1mm}
  \begin{repobox}
    \bfseries\normalsize
      \href{http://github.com/rescience/rescience-submission/article}{\faGithubAlt~Article repository}
  \end{repobox}
      \vspace{-1mm}
  \begin{repobox}
    \bfseries\normalsize
      \href{http://github.com/rescience/rescience-submission/code}{\faGithubAlt~Code repository}
  \end{repobox}
        \hypersetup{urlcolor=blue}
}

\begin{rebox}
\sffamily {\bfseries A reference implementation of}
\small
\begin{flushleft}
\begin{itemize}
    \item[→] Benincà, E., Huisman, J., Heerkloss, R., Jöhnk, K.D., Branco, P., Van
Nes, E.H., Scheffer, M. \& Ellner, S.P. (2008) Chaos in a long-term
experiment with a plankton community. Nature, 451, 822--825 DOI:
10.1038/nature06512
  \end{itemize}\par
\end{flushleft}
\end{rebox}


\section{Introduction}\label{introduction}

The original paper describes analyses of fluctuations in the abundance
of organisms in a plankton community derived from the Baltic Sea, housed
in a laboratory environment. The length of the time series (samples
every few days for 2,300 days) allowed for analyses revealing that the
observed dynamics exhibited characteristics consistent with chaos
produced by non-linear species interactions. The article concludes that
stability is not required for persistence of complex food webs, and that
long-term prediction of abundances may be fundamentally impossible. The
demonstration of chaotic dynamics and limited forecast horizons (sensu
\textcite{Petchey2015}) are important in the field of ecology, since the
ability to predict dynamics is an open question with considerable
applied importance \textcite{Petchey2015} \textcite{Mouquet2015}.

\section{Methods}\label{methods}

This reproduction started with the raw data (source given below) and
used information from the original paper, the supplementary information
\href{http://www.nature.com/nature/journal/v451/n7180/extref/nature06512-s1.pdf}{the
Supplement to the Nature paper}, and communications with Elisa Benincà
and Stephen Ellner. The latter provided code used to produce results in
the original paper, and its use in this reproduction is indicated below.

\subsection{Scope of the reproduction}\label{scope-of-the-reproduction}

An attempt was made to reproduce the majority of the results in the
original article. Instances where we did not attempt to reproduce a
result are detailed below.

\subsection{The data}\label{the-data}

The data are available as an Excel file supplement to
\href{http://onlinelibrary.wiley.com/doi/10.1111/j.1461-0248.2009.01391.x/abstract}{an
Ecology Letters publication} \textcite{Beninca2009}. The Excel file
contains several datasheets. Two were particularly important, as they
are the source of the raw data (one contains original species
abundances, the one with the nutrient concentrations). We saved these
two datasheets as comma separated value (csv) text files. In the code
associated with this reproduction, these data files are read from the
associated github repository.

Another datasheet in the Ecology Letters supplement contains transformed
variables (we also saved this as csv file, in order to use it in this
reproduction). We also received a dataset direct from Steve Ellner, see
below for details.

The original species abundance data contained errors (e.g., a few
numerica values had a comma in place of a period as the decimal
separator) that suggested that this was not the exact version of the
dataset used in the original article, or that this was the exact
dataset, but with errors corrected.

\subsection{Reproduction environment}\label{reproduction-environment}

The R language and environment for statistical computing and graphics
was used to make the reproduction. Additional R packages required are
specified in the code associated with this reproduction.

The code for this reproduction resides in an R markdown document, as
well as a source file containing some required functions. Some of the
code takes several minutes to run, so an intermediate data file is
provided with results from this code.

\section{Results}\label{results}

\subsection{Population dynamics}\label{population-dynamics}

The reproduced populations dynamics were at least very similar to those
in figure 1b-g of the original publication (figure \ref{fig:dynamics}).
Note that we plot fourth root transformed values, rather than raw
abundances with a y-axis break, as in the original article.

\begin{figure}[htbp]
\centering
\includegraphics{figures/obs_pop_dyn.pdf}
\caption{\label{fig:dynamics}Observed population dynamics.}
\end{figure}

\subsection{Data transformation}\label{data-transformation}

The following transformation steps were used:

\begin{enumerate}
\def\labelenumi{\arabic{enumi}.}
\tightlist
\item
  Time series shortened to remove long sequences of zeros.
\item
  Interpolation to create equally spaced observations in time series.
\item
  Fourth root transformation.
\item
  Detrending of five of the time series.
\item
  Rescaling to zero mean and unit standard deviation.
\end{enumerate}

The method for selecting the zeros to remove was unclear. In order to
reproduce this step, we removed the same data as in the original study,
by matching to the transformed data with zeros removed in the Ecology
Letters supplement Excel file mentioned above. All remaining
transformation steps were performed independently of this dataset. Our
reproduction of the transformed data closely matched the published
transformed data.

The data received directly from Stephen Ellner was interpolated, but
without zeros removed. Our interpolated data, without zeros removed,
matched closely this data.

\subsection{Correlations among species
abundances}\label{correlations-among-species-abundances}

Correlations among species abundances presented in Table 1 of the
original article closely matched our reproduced correlations, calculated
from the transformed data with zeros removed (figure
\ref{fig:corr_comp}). Deviations between the original and reproduced
correlations are relatively small (less than 0.072 units) and
infrequent.

\begin{figure}[htbp]
\centering
\includegraphics{figures/correlation_comparison.pdf}
\caption{\label{fig:corr_comp}Comparison of calculated correlations
among species abundances in the original article and this reproduction.}
\end{figure}

Highlighted in the text of the original paper were: negative
correlations of picophytoplankton with protozoa, and of
nanophytoplankton both with rotifers and calanoid copepods, positive
correlation of picophytoplankton with calanoid copepods, negative
correlation between bacteria and ostracods, and positive correlation
between bacteria and phosphorus. All of these correlations were
reproduced.

\subsection{Spectral analyses}\label{spectral-analyses}

Spectral analyses in the original paper were presented graphically in
figures S3 (raw spectrograms) and S4 (Welch periodograms). These graphs
were, apparently, visually inspected to reveal: ``fluctuations covered a
range of different periodicities'', and ``picophytoplankton, rotifers
and calanoid copepods seemed to fluctuate predominantly with a
periodicity of about 30 days.'' It is unclear how these conclusions were
derived from figures S3 and S4 of the original article. Our reproduced
spectra (not shown here, but code provided) were not quantitatively
identical to the spectra in the original article.

\subsection{Lyapunov exponents by direct
method}\label{lyapunov-exponents-by-direct-method}

Reproduced divergence rates (figure \ref{fig:divergence}) and comparison
of the original and reproduced Lyapunov exponents (figure
\ref{fig:LE_comparison}). The original article states: ``the distance
between initially nearby trajectories increased over time, and reached a
plateau after about 20--30 days''. The reproduced results appear not
inconsistent with this statement, except for one group of species
(Harpacticoids). The original article also stated that the analyses
``yielded significantly positive Lyapunov exponents of strikingly
similar value for all species (Fig. 3; mean exponent = 0.057 per day,
s.d. = 0.005 per day, n = 9)''. Reproduced exponents had very similar
mean value, but had about four times greater standard deviation (mean =
0.055 and s.d. = 0.019).

\begin{figure}[htbp]
\centering
\includegraphics{figures/div_rate.pdf}
\caption{\label{fig:divergence}Reproduced divergence rates and Lyapunov
exponents (figure 3 in the original article).}
\end{figure}

\begin{figure}[htbp]
\centering
\includegraphics{figures/LE_comparison.pdf}
\caption{\label{fig:LE_comparison}Comparison of Lyapunov exponents,
estimated by direct method, in the original article and this
reproduction.}
\end{figure}

\subsection{Lyapunov exponents by indirect
method}\label{lyapunov-exponents-by-indirect-method}

The original paper reported global Lyapunov exponent calculated via two
modelling approaches (neural network and generalised additive models
{[}GAMs{]}). Only the GAM approach was reproduced, with the assistance
of code donated by Stephen Ellner. The original article obtained a
global Lyapunov exponent of λ=0.08 day-1. The reproduced value was 0.04.
We did not reproduce the bootstrapping used to give confidence intervals
around this estimate.

\subsection{Predictability decay}\label{predictability-decay}

The article stated: ``For short-term forecasts of only a few days, most
species had a high predictability of R2 = 0.70 -- 0.90 (Fig. 2).
However, the predictability of the species was much reduced when
prediction times were extended to 15--30days.'' The reproduced
predictabilities, which were calculated from the GAMs, were consistent
with these qualitative statements, though were quantitatively different
(\ref{fig:prediction_distance}). We did not reproduce the predictability
estimates for linear models.

\begin{figure}[htbp]
\centering
\includegraphics{figures/prediction_distance.pdf}
\caption{\label{fig:prediction_distance}Predictability (correlation
between predicted and observed abundances) and prediction distance
(days) (figure 2 in the original article).}
\end{figure}

\section{Conclusion}\label{conclusion}

Although we were not able to make a quantitatively accurate reproduction
of all results of the original article, the qualitative results were
largely identical. For example, all Lyapunov exponents estimated by
direct method are positive, as in the original article, consistent with
chaotic dynamics. Quantitative differences may have resulted from
difference in algorithms used. For example,the original used the
\href{http://www.mpipks-dresden.mpg.de/~tisean/}{Tisean software} to
calculate Lyapunov exponents. As this was available from CRAN
\href{http://cran.r-project.org/web/packages/RTisean/index.html}{until
mid 2014} and since it is a bit less well integrated with R, we instead
use the tseriesChaos package \textcite{tseriesChaos}, which in any case
was largely inspired by the TISEAN project. In addition, there may have
been some difference in algorithm parameters, as not all parameters
required by the functions we used were reported in the original ms.
There may also have been some difference in data used for specific
analyses, e.g., data with zeros removed or not, as it was not always
possible to be totally sure the reproduction used exactly the same data
as the original article.

In conclusion, this reproduction supports the general scientific
conclusions of the original article, but also shows how difficult can be
an accurate quantitative reproduction, even in the presence of the
extensive methodological details provided alongside the original
article.

\section{Acknowledgements}\label{acknowledgements}

This reproduction was made as part of the Reproducible Research in
Ecology, Evolution, Behaviour, and Environmental Studies (RREEBES)
Course, lead by Owen Petchey at the University of Zurich. More
information about the course
\href{https://github.com/opetchey/RREEBES/blob/master/README.md}{here}
on github.

{\sffamily \small
  \printbibliography[title=References]
}
\end{document}
