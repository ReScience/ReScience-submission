\documentclass[10pt,a4paper,onecolumn]{article}
% \usepackage[utf8]{inputenc}
\usepackage{marginnote}
\usepackage{graphicx}
\usepackage{xcolor}
\usepackage{authblk,etoolbox}
\usepackage{titlesec}
\usepackage{calc}
\usepackage{hyperref}
\hypersetup{breaklinks=true,
            bookmarks=true,
            pdfauthor=
{
      René Larisch,
      Name Surname,,
  },
            pdftitle=
{
[Re] This is the title
},
            colorlinks=true,
            citecolor=blue,
            urlcolor=blue,
            linkcolor=blue,
            pdfborder={0 0 0}}
\urlstyle{same}
\usepackage{tcolorbox}
\usepackage{ragged2e}
\usepackage{fontspec}
\usepackage{fontawesome}
\usepackage{caption}
\usepackage{listings}
\lstnewenvironment{code}{\lstset{language=Haskell,basicstyle=\small\ttfamily}}{}



%\usepackage{fancyvrb}
%\VerbatimFootnotes
%\usepackage{graphicx}
%\usepackage{mdframed}
%\newmdenv[backgroundcolor=lightgray]{Shaded}


\usepackage{longtable,booktabs}

\usepackage[
  backend=biber,
%  style=alphabetic,
%  citestyle=numeric
]{biblatex}
\bibliography{bibliography.bib}



% --- Macros ------------------------------------------------------------------
\renewcommand*{\bibfont}{\small \sffamily}

\definecolor{red}{HTML}{CF232B}
\newcommand{\ReScience}{Re{\bfseries \textcolor{red}{Science}}}

\newtcolorbox{rebox}
   {colback=blue!5!white, colframe=blue!40!white,
     boxrule=0.5pt, arc=2pt, fonttitle=\sffamily\scshape\bfseries,
     left=6pt, right=20pt, top=6pt, bottom=6pt}

\newtcolorbox{repobox}
   {colback=red, colframe=red!75!black,
     boxrule=0.5pt, arc=2pt, left=6pt, right=6pt, top=3pt, bottom=3pt}

% fix for pandoc 1.14     
\newcommand{\tightlist}{%
  \setlength{\itemsep}{1pt}\setlength{\parskip}{0pt}\setlength{\parsep}{0pt}}

% --- Style -------------------------------------------------------------------
\renewcommand*{\bibfont}{\small \sffamily}
\renewcommand{\captionfont}{\small\sffamily}
\renewcommand{\captionlabelfont}{\bfseries}

\makeatletter
\renewcommand\@biblabel[1]{{\bf #1.}}
\makeatother

% --- Page layout -------------------------------------------------------------
\usepackage[top=3.5cm, bottom=3cm, right=1.5cm, left=1.5cm,
            headheight=2.2cm, reversemp, includemp, marginparwidth=4.5cm]{geometry}

% --- Section/SubSection/SubSubSection ----------------------------------------
\titleformat{\section}
  {\normalfont\sffamily\Large\bfseries}
  {}{0pt}{}
\titleformat{\subsection}
  {\normalfont\sffamily\large\bfseries}
  {}{0pt}{}
\titleformat{\subsubsection}
  {\normalfont\sffamily\bfseries}
  {}{0pt}{}
\titleformat*{\paragraph}
  {\sffamily\normalsize}


% --- Header / Footer ---------------------------------------------------------
\usepackage{fancyhdr}
\pagestyle{fancy}
%\renewcommand{\headrulewidth}{0.50pt}
\renewcommand{\headrulewidth}{0pt}
\fancyhead[L]{\hspace{-1cm}\includegraphics[width=4.0cm]{rescience-logo.pdf}}
\fancyhead[C]{}
\fancyhead[R]{} 
\renewcommand{\footrulewidth}{0.25pt}

\fancyfoot[L]{\hypersetup{urlcolor=red}
              \sffamily \ReScience~$\vert$
              \href{http://rescience.github.io}{rescience.github.io}
              \hypersetup{urlcolor=blue}}
\fancyfoot[C]{\sffamily 1 - \thepage}
\fancyfoot[R]{\sffamily Sep 2015 $\vert$
                        Volume \textbf{1} $\vert$
                        Issue \textbf{1}}
\pagestyle{fancy}
\makeatletter
\let\ps@plain\ps@fancy
\fancyheadoffset[L]{4.5cm}
\fancyfootoffset[L]{4.5cm}

% --- Title / Authors ---------------------------------------------------------
% patch \maketitle so that it doesn't center
\patchcmd{\@maketitle}{center}{flushleft}{}{}
\patchcmd{\@maketitle}{center}{flushleft}{}{}
% patch \maketitle so that the font size for the title is normal
\patchcmd{\@maketitle}{\LARGE}{\LARGE\sffamily}{}{}
% patch the patch by authblk so that the author block is flush left
\def\maketitle{{%
  \renewenvironment{tabular}[2][]
    {\begin{flushleft}}
    {\end{flushleft}}
  \AB@maketitle}}
\makeatletter
\renewcommand\AB@affilsepx{ \protect\Affilfont}
%\renewcommand\AB@affilnote[1]{{\bfseries #1}\hspace{2pt}}
\renewcommand\AB@affilnote[1]{{\bfseries #1}\hspace{3pt}}
\makeatother
\renewcommand\Authfont{\sffamily\bfseries}
\renewcommand\Affilfont{\sffamily\small\mdseries}
\setlength{\affilsep}{1em}

\LetLtxMacro{\OldIncludegraphics}{\includegraphics}
\renewcommand{\includegraphics}[2][]{\OldIncludegraphics[width=12cm, #1]{#2}}


% --- Document ----------------------------------------------------------------
\title{[Re] This is the title}

    \usepackage{authblk}
                        \author[1]{René Larisch}
                    \author[2, 3]{Name Surname,}
                            \affil[1]{Affiliation Dept/Program/Center, Institution Name, City, State, Country}
                    \affil[2]{Affiliation Dept/Program/Center, Institution Name, City, State, Country}
                    \affil[3]{Affiliation Dept/Program/Center, Institution Name, City, State, Country}
            
\date{\vspace{-5mm}
      \sffamily \small \href{mailto:corresponding-author@mail.com}{corresponding-author@mail.com}}


\setlength\LTleft{0pt}
\setlength\LTright{0pt}


\begin{document}
\maketitle

\marginpar{
  %\hrule
  \sffamily\small
  %\vspace{2mm}
  {\bfseries Editor}\\
  Name Surname\\

  {\bfseries Reviewers}\\
        Name Surname\\
        Name Surname\\
  
  {\bfseries Received}  Sep, 1, 2015\\
  {\bfseries Accepted}  Sep, 1, 2015\\
  {\bfseries Published} Sep, 1, 2015\\

  {\bfseries Licence}   \href{http://creativecommons.org/licenses/by/4.0/}{CC-BY}

  \begin{flushleft}
  {\bfseries Competing Interests:}\\
  The authors have declared that no competing interests exist.
  \end{flushleft}

  \hrule
  \vspace{3mm}

  \hypersetup{urlcolor=white}
  
    \vspace{-1mm}
  \begin{repobox}
    \bfseries\normalsize
      \href{http://github.com/rescience/rescience-submission/article}{\faGithubAlt~Article repository}
  \end{repobox}
      \vspace{-1mm}
  \begin{repobox}
    \bfseries\normalsize
      \href{http://github.com/rescience/rescience-submission/code}{\faGithubAlt~Code repository}
  \end{repobox}
        \hypersetup{urlcolor=blue}
}

\begin{rebox}
\sffamily {\bfseries A reference implementation of}
\small
\begin{flushleft}
\begin{itemize}
    \item[→] Original article (title, authors, journal, doi)
  \end{itemize}\par
\end{flushleft}
\end{rebox}


\section{Introduction}\label{introduction}

The introduction should introduce the original paper and put it in
context (e.g.~is it an important paper in the domain ?). You must also
specify if there was an implementation available somewhere and provide a
link to it if relevant (and in such a case, you have to specify if the
proposed replication is based on this original implementation). You
should also introduce your implementation by listing language, tools,
libraries, etc. and motivate choices if relevant.

\section{Methods}\label{methods}

The methods section should explain how you replicated the original
results:

\begin{itemize}
\tightlist
\item
  did you use paper description
\item
  did you contact authors ?
\item
  did you use original sources ?
\item
  did you modify some parts ?
\item
  etc.
\end{itemize}

If relevevant in your domain, you should also provide a new standardized
description of the work.

\section{Results}\label{results}

Results should be compared with original results and you have to explain
why you think they are the same or why they may differ (qualitative
result vs quantitative result). Note that it is not necessary to redo
all the original analysis of the results.

\section{Conclusion}\label{conclusion}

Conclusion, at the very minimum, should indicate very clearly if you
were able to replicate original results. If it was not possible but you
found the reason why (error in the original results), you should exlain
it.

\hypertarget{tbl:table}{}
\begin{longtable}[]{@{}llllll@{}}
\caption{\label{tbl:table}Table caption }\tabularnewline
\toprule
Heading 1 & & & Heading 2 & &\tabularnewline
\midrule
\endfirsthead
\toprule
Heading 1 & & & Heading 2 & &\tabularnewline
\midrule
\endhead
cell1 row1 & cell2 row 1 & cell3 row 1 & cell4 row 1 & cell5 row 1 &
cell6 row 1\tabularnewline
cell1 row2 & cell2 row 2 & cell3 row 2 & cell4 row 2 & cell5 row 2 &
cell6 row 2\tabularnewline
cell1 row3 & cell2 row 3 & cell3 row 3 & cell4 row 3 & cell5 row 3 &
cell6 row 3\tabularnewline
\bottomrule
\end{longtable}

A reference to table ~\ref{tbl:table}. A reference to figure
~\ref{fig:logo}. A reference to equation ~\ref{eq:1}. A reference to
citation \textcite{markdown}.

\begin{figure}
\centering
\includegraphics{rescience-logo.pdf}
\caption{Figure caption}\label{fig:logo}
\end{figure}

\begin{equation} A = \sqrt{\frac{B}{C}} \label{eq:1}\end{equation}

{\sffamily \small
  \printbibliography[title=References]
}
\end{document}
